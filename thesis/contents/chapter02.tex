% !TeX root = ../main.tex
\chapter{背景知識與文獻探討}

    此章將介紹本研究相關的文獻與背景知識,包含超音波麥克風干擾器、噪音消除技術與存取控制。

\section{超音波麥克風干擾器}

    超音波麥克風干擾器的發明旨在解決竊聽的問題,例如:在未經允許的前提下,以隱密的方式紀錄不對外公開的私人對話。
錄音是一種常見的竊聽方式,在智慧型設備普及的現代,由於有能力錄音設備體積很小,例如:錄音筆、智慧手機與智慧手錶等。
攻擊者可以輕鬆地在機密會議和私人談話中,在未經會談參與者允許前提下,透過錄音記錄秘密聲音訊息。
因此,保護機密會議和私人談話,避免遭非允許錄製對於個人語音談話、商業貿易甚至國家安全都非常重要。

    為了解決上述問題,最近的工業產品與學術研究\cite{chen2020wearable}顯示,可以利用超音波產生噪音來干擾錄音。
麥克風的工作原理就像人的耳朵,如果播放可聽見的聲音(20Hz~20kHz),人類的耳朵與和麥克風都將能夠聽到它。
若果播放的聲音是超音波(20kHz以上),此頻率超過人類耳朵的聽力範圍因此聽不到,但可以被麥克風聽到。
基於此原理 Roy 等人提出了一種稱為 BackDoor 的超音波干擾技術,利用麥克風對於超音波的非線性響應輸出特性將噪音注入麥克風。
透過播放特別設計構造的超音波,可以對麥克風執行拒絕服務 (DoS) 攻擊\cite{roy2017backdoor}。
其中若對於麥克風的非線性響應輸出產生的噪音的幅度若大於人聲的幅度,
錄音設備就只能記錄噪音,人與人的對談聲幾乎無法識別\cite{shen2019jamsys}。
以此我們就可以保持會談內容隱私必免遭到錄音竊聽,而解決非法取得有效之聲音記錄的問題。

    為優化超音波麥克風干擾器的效率,Yuxin Chen 等人提出,
透過將超音波麥克風干擾器製造成環形穿戴裝置,可以有效的解決干擾死角的問題\cite{chen2020demonstrating}。
並且解釋了超音波麥克風干擾器對於麥克風的非線性響應輸出其原理為麥克風系統為一複雜的組成的不可知系統,
會將高頻的聲音能量轉換投影至人耳可聽見頻率內,而產生噪音干擾\cite{chen2019understanding}。
另外,Hao Shen 等人也提出於多麥克風超音波干擾器的情境中,
透過改變擺放位置與角度的變化優化超音波麥克風干擾器的效率\cite{shen2019jamsys}。

\section{噪音消除技術}

\section{存取控制}