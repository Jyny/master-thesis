% !TeX root = ../main.tex
\chapter{緒論}\label{chapter:intro}

\section{研究背景動機}\label{section:intro-background}

    本文實作了一套系統,於部署超音波麥克風干擾器的會談情境中,允許特定參與者於會談結束後,得以取得有效之聲音記錄。


\section{研究目的與貢獻}\label{section:intro-purpose}

\section{論文架構}\label{section:intro-arch}

    本論文將研究分為五個章節描述。
首先第一章\nameref{chapter:intro}描述包含\nameref{section:intro-background}、
\nameref{section:intro-purpose}與\nameref{section:intro-arch}。
第二章\nameref{chapter:background}針對本研究所使用相關技術的背景做介紹與整理,
包含\nameref{section:background-jammer}、
\nameref{section:background-anc}與\nameref{section:background-access-control}。
第三張將詳細描述\nameref{chapter:method},其中分成\nameref{sec:symbol}、\nameref{sec:arch}、
\nameref{sec:system-flow}、\nameref{sec:protocol}、\nameref{sec:anc}與\nameref{sec:estimate}。
第四章則針對本研究所設計之系統進行\nameref{chapter:exp},
包括\nameref{sec:impl}、\nameref{sec:exp}與\nameref{sec:analysis}。
最於第五章對本研究成果進行\nameref{chapter:conclusion}。