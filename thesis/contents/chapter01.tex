% !TeX root = ../main.tex
\chapter{緒論}\label{chapter:intro}

\section{研究背景}\label{section:intro-background}

    隨著現代科技快速進步,具有麥克風或錄音功能的智慧型裝置越來越普及與短小輕便,生活中也隨處可見有錄音能力的裝置。
從各式語音助理的普及、智慧城市/住宅的興起,到無時不刻保護我們安全的監控系統,各種具備錄音能力的設備在我們日常中處處可見。
這些裝置除了為我們生活帶來便利性外,也對我們的隱私帶來新的風險,
若這些麥克風遭到惡意利用,在非授權的情況下紀錄人們的聲音,則會造成隱私洩露的危害。

    現代生活中,人們常常需要進行機密談話,例如討論商業策略或先進技術研究等情境。
為保障談話的隱私,常見的做法例如:尋找第三方的公共咖啡廳、共同工作空間的會議包廂或機構內部的會議室,
透過尋求前述場境作為隱蔽或半開放空間,來隔絕偷聽者的存在。
但在智慧裝置隨處可見隨手可得的時代,我們不能忽略隱藏在生活週遭的麥克風帶來的潛在風險。
考慮到惡意使用的情況,如有存在惡意竊聽的第三方,預先在會議的空間內放置隱藏式的錄音裝置,
此舉實質上是在破壞會議參與者對於空間的信任。
若會議的參與者們不希望會議或談話過程被紀錄,通常會藉由限制隨身電子設備的使用來達到不被記錄的目的,
而若會議的內容秘密等級較高,會進一步透過建立機密資訊隔離設施
(Sensitive compartmented information facility, SCIF) 來保障秘密會議的需求。
例如:機密資訊隔離設施具有有聲音隔離的能力,或透過人工檢查來阻止與會者攜入電子設備。

    根據保密等級不同,建置合規的機密資訊隔離設施的成本相當可觀,
包含額外的人力、硬體設施、信任的第三方以及嚴謹的作業流程等。
且若會議的過程中若有聲音紀錄的需求,需要會談參與者共同信任錄音裝置或會談聲音紀錄的保存與散佈程序。
由此可知,因智慧型裝置與微型麥克風的成本降低與普及,大大增加了秘密保護的成本與未知洩漏的風險。
因此本論文針對上述問題進行研究,提出一套解決方案,
使會談參與者在不信任會議空間或其他會談參與者的情況下得以安全的進行秘密會談。
並且本研究提出的解決方案可以在在秘密會談進行的同時紀錄會談的聲音紀錄內容,
且會談聲音紀錄不會有未授權存取,保有其機密性。


\section{研究目的與貢獻}\label{section:intro-purpose}

    本研究之目的為,建構一套聲音記錄及存取系統,
在保障會談隱私的同時,也允許特定會談參與者,得以於私密會談結束後取得會談聲音記錄。
本研究基於麥克風干擾與主動噪音消除技術,透過超音波對麥克風造成的非線性響應輸出與自適應噪音消除高關聯度噪音等特性,
開發一套干擾與還原的機制。同時藉由密碼系統的機密性與不可否認性,發展一套以其為基礎的會談聲音記錄存取協定。

本研究之貢獻為,透過麥克風干擾\cite{chen2020wearable}\cite{roy2017backdoor},
將其發展成用於保障會談隱私的電子聲音屏障。
與消除特定模式超音波\cite{he2019canceling},發展成用於還原被干擾的聲音。
兩者結合且提出一套用於干擾保密與解密還原的機制並實作出原型。
實驗方面,透過不同會談聲音內容與干擾噪音的信噪比,評估干擾保密與解密還原的效用。
安全性分析方面,將此機制視為一串流加解密演算法進行安全性分析,
討論於密碼學領域上的常見攻擊手法對於此機制是否產生威脅。
接著延伸前述干擾保密與解密還原的機制,發展一套會談聲音存取控制協定且實作出原型。
並且對此系統進行效能評估與協定安全性分析。
本研究所提出之「干擾保密與解密還原」與「談聲音存取控制協定」的方法,
相較於 Lingkun Li 等人研究成果 \cite{li2020patronus},
降低實作的複雜度且提升系統可用性符合聲音記錄及存取情境。


\section{論文架構}\label{section:intro-arch}

    本論文將研究分為五個章節描述。
首先第一章\nameref{chapter:intro}描述包含\nameref{section:intro-background}、
\nameref{section:intro-purpose}與\nameref{section:intro-arch}。
第二章\nameref{chapter:background}針對本研究所使用相關技術的背景做介紹與整理,
包含\nameref{section:background-jammer}、
\nameref{section:background-anc}與\nameref{section:background-access-control}。
第三張將詳細描述\nameref{chapter:method},其中分成\nameref{sec:symbol}、\nameref{sec:arch}、
\nameref{sec:system-flow}、\nameref{sec:protocol}、\nameref{sec:anc}與\nameref{sec:estimate}。
第四章則針對本研究所設計之系統進行\nameref{chapter:exp},
包括\nameref{sec:impl}、\nameref{sec:exp}與\nameref{sec:analysis}。
最於第五章對本研究成果進行\nameref{chapter:conclusion}。