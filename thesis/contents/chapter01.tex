% !TeX root = ../main.tex
\chapter{緒論}\label{chapter:intro}

\section{研究背景動機}\label{section:intro-background}


\section{研究目的與貢獻}\label{section:intro-purpose}

    本研究之目的為,建構一套聲音記錄及存取系統,
在保障會談隱私的同時,也允許特定會談參與者,得以於私密會談結束後取得會談聲音記錄。
本研究基於麥克風干擾與主動噪音消除技術,透過超音波對麥克風造成的非線性響應輸出與自適應噪音消除高關聯度噪音等特性,
提出並實作一套干擾與還原的機制。同時藉由密碼系統的機密性與不可否認性,發展且實作一套以其為基礎的會談聲音記錄存取協定。

本研究之貢獻為,基於目前技術發展現況,如超音波麥克風干擾技術的研究成果,將其實作為應用於保障會談隱私的電子聲音屏障。
與另一技術,如消除特定模式超音波,用於還原被干擾的聲音。
接著對此系統加以實驗,比較於不同信噪比的環境下,干擾保密與解密還原的效用評估。
並將此方法視為串流加解密演算法進行安全性分析,討論於密碼學領域上的常見攻擊手法對於此系統的影響。
接著利用前述干擾保密與解密還原的方法,發展一套會談聲音存取控制協定,並且對此系統進行效能評估與協定安全性分析。


\section{論文架構}\label{section:intro-arch}

    本論文將研究分為五個章節描述。
首先第一章\nameref{chapter:intro}描述包含\nameref{section:intro-background}、
\nameref{section:intro-purpose}與\nameref{section:intro-arch}。
第二章\nameref{chapter:background}針對本研究所使用相關技術的背景做介紹與整理,
包含\nameref{section:background-jammer}、
\nameref{section:background-anc}與\nameref{section:background-access-control}。
第三張將詳細描述\nameref{chapter:method},其中分成\nameref{sec:symbol}、\nameref{sec:arch}、
\nameref{sec:system-flow}、\nameref{sec:protocol}、\nameref{sec:anc}與\nameref{sec:estimate}。
第四章則針對本研究所設計之系統進行\nameref{chapter:exp},
包括\nameref{sec:impl}、\nameref{sec:exp}與\nameref{sec:analysis}。
最於第五章對本研究成果進行\nameref{chapter:conclusion}。