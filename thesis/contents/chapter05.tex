% !TeX root = ../main.tex
\chapter{總結}\label{chapter:conclusion}

    隨著現代科技快速進步,具有麥克風或錄音功能的智慧型裝置越來越普及與短小輕便,同時也使得非授權的錄音變得難以預防。
從各式語音助理的普及、智慧城住宅的興起等,各種具備錄音能力的設備在我們日常中處處可見。
這些裝置除了為我們生活帶來便利性外,也對我們的隱私帶來新的風險,
若這些麥克風遭到惡意利用,在非授權的情況下紀錄人們的聲音,則會造成隱私洩露的危害。
為保障談話的隱私,可以透過尋求前述場境作為隱蔽或半開放空間,來隔絕偷聽者的存在。
但在智慧裝置隨處可見隨手可得的時代,我們不能忽略隱藏在生活週遭的麥克風帶來的潛在風險。
如有存在惡意竊聽的第三方,預先在會議的空間內放置隱藏式的錄音裝置,此舉實質上是在破壞會議參與者對於空間的信任。
若會談的過程中若有聲音紀錄的需求,同時也需要會談參與者共同信任錄音裝置或會談聲音紀錄的保存與散佈程序。
由此可知,智慧型裝置與微型麥克風的成本降低與普及,大大增加了秘密保護的成本與未知洩漏的風險。
因此,如何建立一個保障隱私的對話環境,以防止未授權的錄音或竊聽,將會是目前技術面臨的挑戰及目標。

    本論文針對上述問題進行研究,提出一套解決方案。
在保障會談隱私的目標下,滿足當會談參與者在不信任會議空間或其他會談參與者時,
得以安全的進行秘密會談與記錄會談聲音內容。
且會談結束後若有存取會談聲音紀錄的需求,可以保障在未授權存取的前提下,會談聲音紀錄保有機密性。

    透過最近的研究可以得知,超音波對於麥克風的非線性響應輸出可用來干擾錄音中的麥克風。
當原始錄音與干擾噪音的信噪比達到一定比例時,等同於對麥克風的錄音功能進行阻斷服務攻擊(DoS),
其對於錄音結果的影響是毀滅性的。若在秘密對談的場域內部署超音波干擾器,
此舉可以解決被未授權第三方竊聽或錄音的問題,但同時也帶來新的副作用,使得場域所有麥克風因此失去功能。
因此本研究提出,將麥克風干擾進一步延伸與主動噪音消除技術結合,來解決上述問題。
透過麥克風干擾,將其發展成用於保障會談隱私的電子聲音屏障。
與消除特定模式超音波,發展成用於還原被干擾的聲音。
在還原干擾的過程中,為增加噪音消除的效果,本研究加入離散時間誤差推估演算法,
用於對齊修正輸入噪音聲音紀錄與欲消去噪音的離散時間誤差。
透過上述技術結合,提出一套用於干擾保密與解密還原的機制並實作出原型。

    接著藉透過非對稱是加密系統與金鑰分割分享秘密的方法,發展一套以其為基礎的會談聲音記錄存取協定。
協定中針對會談情境定義多個角色,包含會談擁有者、會談終端與與解封伺服器等。
協定的流程包含會談生命週期不同階段如:\nameref{subsec:initialize}、
\nameref{subsec:sessioning}與\nameref{subsec:unseal}。
透過本研究所提出會談聲音記錄存取協定使得特定會談參與者,得以於私密會談結束後取得還原的有效之聲音記錄。

    實驗方面,針對本研究所設計之系統與實作進行驗證。
透過設計不同音量的干擾噪音,創造一不同會談聲音內容與干擾噪音的信噪比的環境,
近一步評估干擾保密與解密還原在不同信噪比環境的的效用。
評估的方法為使虛擬客觀語音品質估測者(ViSQOL),
根據參考未干擾的會談聲音紀錄與被干擾的會談聲音紀錄,
產生聲音品質聽感的平均意見分數 MOS-LQO,從一到五評分客觀語音品質與實際人耳聽感描述,進行評估。
最後結合干擾保密與解密還原與會談聲音存取控制協定的兩者,進行效能評估,
評估分析當還原會談聲音記錄時,封伺服器執行聲音樣本離散時間推估演算法與主動式噪音消除的時間複雜度性。

    安全性分析方面,本研究將干擾保密與解密還原機制,視為一串流加解密演算法進行安全性分析,
討論於密碼學領域上的常見攻擊手法對於此機制是否產生威脅。
包含唯密文攻擊、選擇明文攻擊與選擇密文攻擊。
接著結合干擾保密與解密還原與會談聲音存取控制協定的兩者,針對通訊層面可能的攻擊,
或協定在金鑰的脆弱點,對此系統進行效能評估與協定安全性分析。