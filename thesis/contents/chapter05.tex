% !TeX root = ../main.tex
\chapter{總結}\label{chapter:conclusion}

    隨著現代科技快速進步,具有麥克風或錄音功能的智慧型裝置越來越普及與短小輕便,同時也使得非授權的錄音變得難以預防。
從各式語音助理的普及、智慧城住宅的興起等,各種具備錄音能力的設備在我們日常中處處可見。
這些裝置除了為我們生活帶來便利性外,也對我們的隱私帶來新的風險,
若這些麥克風遭到惡意利用,在非授權的情況下紀錄人們的聲音,則會造成隱私洩露的危害。
為保障談話的隱私,可以透過尋求前述場境作為隱蔽或半開放空間,來隔絕偷聽者的存在。
但在智慧裝置隨處可見隨手可得的時代,我們不能忽略隱藏在生活週遭的麥克風帶來的潛在風險。
如有存在惡意竊聽的第三方,預先在會議的空間內放置隱藏式的錄音裝置,此舉實質上是在破壞會議參與者對於空間的信任。
若會談的過程中若有聲音紀錄的需求,同時也需要會談參與者共同信任錄音裝置或會談聲音紀錄的保存與散佈程序。
由此可知,智慧型裝置與微型麥克風的成本降低與普及,大大增加了秘密保護的成本與未知洩漏的風險。
因此,如何建立一個保障隱私的對話環境,以防止未授權的錄音或竊聽,將會是目前技術面臨的挑戰及目標。

    本論文針對上述問題進行研究,提出一套解決方案,使會談參與者在不信任會議空間或其他會談參與者的情況下,
得以安全的進行秘密會談與記錄會談聲音內容。且保障在尚未授權存取的前提下,會談聲音紀錄保有機密性。
