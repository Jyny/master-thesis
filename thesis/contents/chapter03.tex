\chapter{系統架構與實作}

    本研究所設計之系統,利用超聲波對麥克風造成的非線性響應輸出,與主動式噪音控制消除高關聯度噪音等特性。
嘗試解決在部署麥克風干擾器的場域裡,無法取得有效聲學紀錄的問題。
同時借鏡密碼系統的機密性與不可否認性,發展一套於會談結束後,以其為基礎的會談錄音存取控制機制。
透過上述機制的結合,能夠使在部署超音波麥克風干擾器的會談情境中,於會談結束後,允許特定參與者得以取得有效之聲音記錄。


\section{系統架構}

\begin{figure}[H]
    \centering
    \includegraphics[width=0.8\textwidth]{single-owner-architecture}
    \caption{單一 Owner 系統架構圖}
    \label{fig.s-o-arch}
\end{figure}

    本系統由 MeetingBox、Server 以及會談參與者 Attender 所組成。
Attender 於 MeetingBox 的周圍進行會談 ({\it Meeting Session}),
MeetingBox 於會談進行間開啟超聲波麥克風干擾器,
因此鄰近的麥克風或周邊的聲音記錄裝置,都將因受到干擾而失效,Attender 無法自行有效記錄會談的聲音內容。

    MeetingBox 於會談進行中持續紀錄聲音,內容為受到超聲波麥克風干擾器干擾的會談錄音 ($REC_{J}$)。
於會談結束後,受超聲波麥克風干擾器干擾的會談聲音記錄 ($REC_{J}$),將作為系統機制「解封」({\it Unseal}) 的輸入,
使會談參與者 Attender 中的會談擁有者 Owner,獲得前述機制所產生的有效會談聲音記錄 ($REC_{rev}$)。

    各會談的 Attender 僅定義於當次會談,每次參與會談的 Attender 可為不同群體。
Attender 透過 MeetingBox 上的控制介面與其互動,進而控制本研究所設計系統的生命週期。


\subsection{Attender}

    會談參與者 Attender 為一集合,代表所有當次會談 ({\it Meeting Session}) 的參與者。
Attender 包含 Owner,為 Owner 的超集合。其餘非 Owner 的 Attender 視為 not Owner Attender。
Attender 可以透過 MeetingBox 上的控制介面與其互動,來控制會議的開始與終止,本研究以實體按鈕為例。

    Attender 受到 MeetingBox 中的超聲波麥克風干擾器的干擾,使得隨身的麥克風或聲音記錄裝置,
都將因受到干擾而失效,無法有效記錄會談內容。例如:智慧型手機、智慧手錶、筆電、平板、智慧喇叭等。


\subsection{Owner}

\begin{figure}[H]
    \centering
    \includegraphics[width=0.8\textwidth]{multi-owner-architecture}
    \caption{多 Owner 系統架構圖}
    \label{fig.m-o-arch}
\end{figure}

    會談擁有者 Owner 屬於 Attender,為 Attender 的子集合,
Owner 定義為 Attender 中的特權角色。Owner 如未執行本研究所設計的系統機制「註冊會議擁有者」 (Register Owner),
則視為 not Owner Attender。 Owner 於會談 ({\it Meeting Session}) 結束後,
有能力決定是否執行系統機制「解封」 ({\it Unseal}),並獲取 {\it Unseal} 所產生的有效會談聲音記錄 ($REC_{rev}$)。

    Owner 可以是一或多人。當會談中只有一個 Owner 時,為 {\it Single Owner Meeting Session} 情境,
系統架構如圖 \ref{fig.s-o-arch}。當會談中多於一個 Owner 時,為 {\it Multi Owner Meeting Session} 情境,
系統架構如圖 \ref{fig.m-o-arch} 所示。

    Owner 持有智慧型裝置,透過其與 MeetingBox 互動,獲取 MeetingBox 上的資訊,並有能力於會談進行中、會談結束後,
與 Server 持續通訊,執行本研究所設計的系統機制「註冊會議擁有者」 (Register Owner) 與「解封」 ({\it Unseal})。

    上述 Owner 獲取 MeetingBox 資訊的方法,本研究以掃描 QR Code 為例。


\subsection{MeetingBox}

    MeetingBox 為本研究所設計系統核心組成要件的終端,部署於會談 ({\it Meeting Session}) 的場域,
與會議參與者 Attender 實體接觸。透過與 Attender 的互動,成為觸發會談生命週期改變事件的控制核心。
同時也作為系統主要資料 ($REC_{J}$、$REC_{N}$) 的輸入來源。
在本研究的設計中,一組 MeetingBox 僅能同時間處理一場會談的進行。

    本研究所設計的 MeetingBox 裝置包含下列組件:

    \begin{enumerate}
        \item 超音波麥克風干擾器:\\
            使用偽隨機數產生亂數頻率的超音波,於會談進行中開啟。用於干擾鄰近周圍含有麥克風的裝置,使其無法有效記錄會談內容。

        \item 錄音麥克風:\\
            用於紀錄會談的聲音內容 ($REC_{J}$),與產生超聲波麥克風干擾器於麥克風的響應輸出 ($REC_{N}$)

        \item 物理控制介面:\\
            提供 Attender 操作與 MeetingBox 互動,獲得外部觸發事件。本研究實做以實體按鈕為例。

        \item 人機互動介面:\\
            用於傳遞會談的 meta data 與系統狀態提示給 Attender,本研究實做以 QR Code 搭配螢幕顯示為例。

        \item 運算控制核心與網路介面:\\
            周邊裝置控制與邏輯運算核心,用於加密,與 Server 通訊等運算工作。
    \end{enumerate}


\subsection{Server}

    Server 為本研究所設計的系統核心組成要件之一,其設計能同時間處理一場以上的會談 ({\it Meeting Session}) 的進行,
其角色目標為執行系統機制「解封」 ({\it Unseal})。此機制觸發於會談結束後,Owner 向 Server 發送請求執行。
目的為取得有經系統機制 {\it Unseal} 所產生有效之會談聲音記錄 $REC_{rev}$。本研究所設計系統可以確保 Server 在會談開始後,
直到成功解封之前,會談錄音 $REC_{N}$ 保有機密性。

    在系統機制「解封」 ({\it Unseal}) 中,包含第一步:「{\it Challenge}」,與第二步:「{\it ANC}」。
執行時,系統首先透過機制 {\it Challenge},鑑別 Owner。在 {\it Single Owner Meeting Session} 情境中,
Owner 執行 {\it Challenge} 成功之後,Server 即獲得授權,便可以接續執行 {\it Unseal} 機制中的第二步 {\it ANC}。
在 {\it Multi Owner Meeting Session} 的情境中,則等待所有 Owner 執行 {\it Challenge} 成功之後,
Server 才獲得授權,接續執行 {\it ANC}。


\section{符號定義}

    本研究設計之系統所使用符號與其意義如表 \ref{table:tab.symbol} 所示。

\begin{table}[H]
    \centering
    \caption{符號定義表}
    \label{table:tab.symbol}
    \begin{tabular}{ c c }
        \hline
        \bf{符號} & \bf{釋義} \\
        \hline
        $Attender_{m}$ & 會議參與者 Attender 中,任意特定一位 \\
        $Owner_{n}$    & 會議參與者 Owner 中,任意特定一位 \\
        $REC_{J}$      & 已被超聲波麥克風干擾器干擾的會談之聲音記錄 \\
        $REC_{N}$      & 超聲波麥克風干擾器於麥克風的響應輸出之聲音記錄 \\
        $REC_{rev}$    & 執行系統機制 {\it ANC} 所產生有效之會談聲音記錄 \\
        $time_{Max}$   & 會談進行時間長度的最大值、$REC_{N}$ 的時間長度 \\
        $EK$           & 用於當次會談對稱式加密 $REC_{N}$ 的金鑰 \\
        $PK_{n}$       & $Attender_{n}$ 的公開金鑰 \\
        $SK_{n}$       & $Attender_{n}$ 的私密金鑰 \\

    \end{tabular}
\end{table}

\section{系統流程}

    本研究所設計之系統,其流程跟隨會談 ({\it Meeting Session}) 的生命週期,將其分為三個階段說明。
    \begin{enumerate}
        \item 初始化會談 (Initialize {\it Meeting Session}) \\
        \item 進行會談 (Running {\it Meeting Session}) \\
        \item 解封 ({\it Unseal}) \\
    \end{enumerate}

\section{系統機制}

\subsection{錄音存取控制與解密}

\subsection{主動式噪音控制}

\subsection{樣本對齊}

\section{系統實作}