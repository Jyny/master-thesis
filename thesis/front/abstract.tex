% !TeX root = ../main.tex

\begin{abstract}

    隨著科技快速發展,智慧型設備的計算能力提升、體積逐漸縮小與輕量化,使得隨手記錄聲音變得越來越容易。
然而,在錄音設備更輕便的同時,也使得非授權的錄音變得難以預防。
如果非公開談話被非法錄音或擷取,則可能會嚴重侵犯個人隱私。
而科技的發展除了便利性外必須兼顧隱私的保障,因此隨著人們對隱私的更加注重,
如何防止未經授權的錄音,勢必是需要被重視的議題。
綜上所述,如何建立一個保障隱私的對話環境,來被免被未授權的第方三竊聽或錄音,
將會是目前技術面臨的挑戰及目標。

    最近的研究表明,超音波對於麥克風的非線性響應輸出可用來干擾錄音中的麥克風。
當原始錄音與干擾噪音的信噪比達到一定比例時,等同於對麥克風的錄音功能進行阻斷服務攻擊(DoS),
其對於錄音結果的影響是毀滅性的。若在秘密對談的場域內部署超音波干擾器,
此舉可以解決被未授權第三方竊聽或錄音的問題,但同時也帶來新的副作用,使得場域所有麥克風因此失去功能。
例如:欲對秘密會談進行聲音紀錄作為證據保存,亦或想在會談結束後根據錄聲音紀錄建立逐字稿等情境,
皆因部署麥克風干擾器所帶來的副作用而無法實現。

    因此本研究基於麥克風干擾與主動噪音消除技術,
透過超音波對麥克風造成的非線性響應輸出與自適應噪音消除高關聯度噪音等特性,提出一套干擾/還原的方法。
同時藉由密碼系統的機密性與不可否認性,發展一套以其為基礎的會談聲音記錄存取協定。
透過上述技術的結合,解決部署有麥克風干擾器的場域裡,無法取得有效聲學紀錄的問題。
綜上所述,本研究建構了一套聲音記錄及存取系統,在保障會談隱私的同時,
也允許特定參與者,得以於私密會談結束後取得還原的有效之聲音記錄。

\end{abstract}


\begin{abstract*}

    With the advance of science and technology,
the computing power of smart devices has increased,
and the size getting more compact,
these make people easier to record sounds.
However, as the recording equipment becomes easier to carry with,
it also makes eavesdropping becomes more difficult to prevent.
If a private conversation is illegally recorded or captured,
personal privacy may be seriously violated.
The development of science and technology must take account of privacy and convenience.
Therefore, as people emphasize personal privacy,
how to prevent unauthorized recording is bound to be an issue that needs to be taken seriously.
In summary, how to establish a privacy-guaranteed environment
to avoid being eavesdropped on or unauthorized recorded
will be the challenge and goal of modern technology.

    Recent works have demonstrated that the non-linear impulse response of ultrasonic
has the ability to interferes with the microphone while recording.
When the signal-to-noise ratio between the original recording
and the interference noise reaches a certain level,
this is equivalent to a denial of service (DoS) attack on the microphone's recording,
and the impact on the recording result is devastating.
If an ultrasonic jammer is deployed in the chamber of private conversation,
this can solve the problem of eavesdropping or unauthorized recording,
but it also affects all microphones in the same area. For example,
if you want to record a secret meeting as evidence,
or you want to create a verbatim script based on the recording after the meeting.
Thus it's unable to achieve because of the interference of the ultrasonic microphone jammer.

    Therefore, we propose a jamming/de-jamming method
based on microphone jammer and Active Noise Cancellation (ANC) technology.
Utilize the non-linear impulse response of ultrasonic to a jamming microphone,
and de-jammed by active noise canceling the high-correlation noise.
At the same time, we developed a sound record accessing protocol
base on the confidentiality and non-repudiation of the cryptographic.
Through the combination of the above technologies,
solve the problem that acoustic records were corrupted and not accessible
after the private conversation in the area where microphone jammers are deployed.
In summary, we engineered a sound recording and accessing system
that protects the privacy of conversations,
but also makes it possible of allowing authorized participants
to accessing de-jammed records after the private conversation.

\end{abstract*}