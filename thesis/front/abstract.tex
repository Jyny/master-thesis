% !TeX root = ../main.tex

\begin{abstract}

    隨著科技快速發展,各種智慧型設備/個人計算終端也越來越普及,聲音記錄過程也變得越來越易。
然而,隨著聲音紀錄設備更加易於攜帶的同時,也使得非授權的錄音變得難以預防。
如果非公開談話被非法錄音或擷取,則可能會嚴重侵犯個人隱私。
而科技的發展除了便利性外必須兼顧隱私的保障,因此隨著人們對個人隱私的更加注重,
如何防止未經授權的錄音,勢必是需要被重視的議題。
綜上所述,如何建立一個保障隱私的對話環境,來被免被未授權的第方三竊聽或錄音,
將會是目前技術面臨的挑戰及目標。

    最近的研究表明,透過超音波對於麥克風的非線性響應輸出來干擾錄音中的麥克風。
當原始錄音與干擾噪音的信噪比達到一定比例時,等同於對麥克風的錄音功能進行阻斷服務攻擊(DoS),
其對於錄音結果的影響是毀滅性的。若在秘密對談的場域內部署超音波干擾器,
此舉可以解決被未授權第三方竊聽或錄音的問題,但同時也帶來新的副作用,使得場域所有麥克風因此失去功能。
例如:欲對秘密會談進行聲音紀錄作為證據保存,亦或想在會談結束後根據錄聲音紀錄建立逐字稿等情境,
皆因部署麥克風干擾器後帶來的的副作用而無法實現。

    因此本研究提出,基於麥克風干擾與主動噪音消除技術,建立一套聲音記錄及存取系統。
利用超音波對麥克風造成的非線性響應輸出與自適應噪音消除高關聯度噪音等特性,
同時藉由密碼系統的機密性與不可否認性,發展一套以其為基礎的會談聲音記錄存取機制。
透過上述技術的結合,使在部署超音波麥克風干擾器的會談場域裡,允許特定參與者得以取得有效之聲音記錄。
綜上所述,本研究基於超音波麥克風干擾器來保障對話隱私的同時,
解決部署有麥克風干擾器的場域裡,無法取得有效聲學紀錄的問題。

\end{abstract}


\begin{abstract*}

    With the advance of science and technology,
various smart devices/personal computing terminals are becoming more and more popular,
and sound recording has also become easier then ever.
However, as the recording equipment becomes easier to carry with,
eavesdropping becomes difficult to prevent.
If a private conversation is illegally recorded or captured,
personal privacy may be seriously violated.
The development of science and technology must take into account of privacy and convenience.
Therefore, as people emphasis on personal privacy,
how to prevent unauthorized recording is bound to be an issue that needs to be taken seriously.
In summary, how to establish a privacy-guaranteed environment
to avoid being eavesdropped or unauthorized recording
will be the challenge and goal of modern technology.

    Recent works have demonstrated that the non-linear impulse response of ultrasonic
has ability to interferes with the microphone while recording.
When the signal-to-noise ratio between the original recording
and the interference noise reaches a certain ratio,
this is equivalent to a denial of service (DoS) attack on the microphone's recording,
and the impact on the recording result is devastating.
If an ultrasonic jammer is deployed in the chamber of private conversation,
this can solve the problem of eavesdropping or unauthorized recording,
but it also affect all microphones in the same area. For example,
if you want to record a secret meeting as evidence,
or you want to create a verbatim script based on the recording after the meeting.
Thus it's unable to achieve because the interference of ultrasonic microphone jammer.

    Therefore, we proposes a sound recording and accessing system
based on microphone jammer and Active Noise Cancellation (ANC) technology.
Utilize the non-linear impulse response of ultrasonic in to recording
and the adaptive noise canceling of high-correlation noise.
At the same time, we developed a sound record accessing mechanism
base on the confidentiality and non-repudiation of the cryptographic system.
Through the combination of the above technologies,
it make possible of allowing authorized participants to accessing unjammed records
in the area of private conversation, where ultrasonic microphone jammers are deployed.
In summary, we engineered a sound recording and accessing system
based on ultrasonic microphone jammers to keep privacy of conversations,
and solves the problem that acoustic records were corrupted and not accessible
after private conversation in area where microphone jammers are deployed.

\end{abstract*}